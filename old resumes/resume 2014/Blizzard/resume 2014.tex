%%%%%%%%%%%%%%%%%%%%%%%%%%%%%%%%%%%%%%%%%
% Medium Length Graduate Curriculum Vitae
% LaTeX Template
% Version 1.1 (9/12/12)
%
% This template has been downloaded from:
% http://www.LaTeXTemplates.com
%
% Original author:
% Rensselaer Polytechnic Institute (http://www.rpi.edu/dept/arc/training/latex/resumes/)
%
% Important note:
% This template requires the res.cls file to be in the same directory as the
% .tex file. The res.cls file provides the resume style used for structuring the
% document.
%
%%%%%%%%%%%%%%%%%%%%%%%%%%%%%%%%%%%%%%%%%

%----------------------------------------------------------------------------------------
%	PACKAGES AND OTHER DOCUMENT CONFIGURATIONS
%----------------------------------------------------------------------------------------

\documentclass[margin, 10pt]{res} % Use the res.cls style, the font size can be changed to 11pt or 12pt here

\usepackage{latexsym, amssymb, bbm, amsmath}

\usepackage{helvet} % Default font is the helvetica postscript font
%\usepackage{newcent} % To change the default font to the new century schoolbook postscript font uncomment this line and comment the one above

\setlength{\textwidth}{5.1in} % Text width of the document

\begin{document}

%----------------------------------------------------------------------------------------
%	NAME AND ADDRESS SECTION
%----------------------------------------------------------------------------------------

\moveleft.5\hoffset\centerline{\large\bf Allen Wu} % Your name at the top
 
\moveleft\hoffset\vbox{\hrule width\resumewidth height 1pt}\smallskip % Horizontal line after name; adjust line thickness by changing the '1pt'
 
\moveleft.5\hoffset\centerline{1009 E. 57$^{\text{th}}$ Street, Chicago, IL 60637} % Your address
\moveleft.5\hoffset\centerline{lhefriel@gmail.com}
\moveleft.5\hoffset\centerline{(505) 920-4664}

%----------------------------------------------------------------------------------------

\begin{resume}

%----------------------------------------------------------------------------------------
%	OBJECTIVE SECTION
%----------------------------------------------------------------------------------------
% 
%\section{OBJECTIVE}  
%
%Stuff. 

%----------------------------------------------------------------------------------------
%	EDUCATION SECTION
%----------------------------------------------------------------------------------------

\section{EDUCATION}

\textbf{University of Chicago}, Chicago, IL \\
{\sl Bachelor of Arts,} Mathematics, expected Spring 2015 \\
GPA: 3.89/4.00 \\
Relevant Coursework: Honors Calculus I-III, Analysis in $\mathbb{R}^n$ I-III, Honors Elements of Economic Analysis I, Honors Basic Algebra I, Basic Complex Variables \\\\
\textbf{Los Alamos High School}, Los Alamos, NM \\
GPA: 4.33/4.00 \\
Relevant Coursework: AP Computer Science
 
%----------------------------------------------------------------------------------------
%	PROFESSIONAL EXPERIENCE SECTION
%----------------------------------------------------------------------------------------
 
\section{EXPERIENCE}

{\sl Intern} \hfill Summer 2014 \\
\textbf{Sandia National Laboratory}, Resilience and Regulatory Effects

\begin{itemize} \itemsep -2pt % Reduce space between items
\item Performed research regarding new developments in economic modeling and offered input regarding the efficacy of various models. 
\item Investigated, acquired, and organized publicly available data sets. 
\item Wrote code that filtered, analyzed, and consolidated that data, then quickly and iteratively produced relevant graphics. 
\item Reviewed peer papers and proposals for publication. 
\end{itemize}
 
{\sl Student} \hfill Summer 2012 \\
\textbf{Undergraduate Mathematics REU}, University of Chicago 
\begin{itemize} 
\item Attended introductory lectures to higher mathematics and occasionally specialized lectures on specific topics.
\item Researched basic number theory under the guidance of a graduate student mentor.  
\end{itemize} 

{\sl Intern} \hfill Summer 2010 \\
\textbf{Los Alamos National Laboratory}, T-Division
\begin{itemize}
\item Wrote a simple but highly modular program in Java that read data from input arrays and interpolated density graphs according to finite element methodology. The program rotated, deformed, and translated systems of particles in one and two dimensions.
\item Adjusted program specifications and inputs according to mathematical theories to test the boundaries of the methodology and attempt to fix some its more glaring flaws, specifically regarding collision modeling. 
\item Researched the mathematical foundations of the method and provided input on remedies where appropriate. 
\end{itemize} 

%----------------------------------------------------------------------------------------
%	SKILLS SECTION
%----------------------------------------------------------------------------------------

\section{SKILLS} 

{\sl Programming:} 
Python, C++, Java, R, Stata \\
{\sl Software:} 
Excel, OpenOffice \\
%{\sl Languages:} 
%rudimentary Chinese and German \\

%----------------------------------------------------------------------------------------
%	INTERESTS SECTION
%----------------------------------------------------------------------------------------

\section{INTERESTS} 

Board and card games, writing, movies, literary phylogenetics, Louis C.K., Miranda July, the Mountain Goats

%----------------------------------------------------------------------------------------
%	COMMUNITY SERVICE SECTION
%---------------------------------------------------------------------------------------- 
%
%\section{COMMUNITY \\ SERVICE}
%
%Organized and directed the 1988 and 1989 Grand Marshall Week \\
%``Basketball Marathon.'' A 24 hour charity event to benefit the Troy Boys Club. Over 250 people participated each year. 
%
%----------------------------------------------------------------------------------------
%	EXTRA-CURRICULAR ACTIVITIES SECTION
%----------------------------------------------------------------------------------------
%
%\section{EXTRA-CURRICULAR \\ ACTIVITIES} 
%
%Elected {\it House Manager}, Rho Phi Sorority \\
%Elected {\it Sports Chairman} \\
%Attended Krannet Leadership Conference \\
%Headed delegation to Rho Phi Congress \\
%Junior varsity basketball team \\
%Participant, seven intramural athletic teams 

%----------------------------------------------------------------------------------------

\end{resume}
\end{document}